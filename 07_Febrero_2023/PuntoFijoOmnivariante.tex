\documentclass[12pt]{article}
\begin{document}
\begin{flushright}
    07 de Febrero, 2023 \linebreak
    Aguilar Sánchez Diego Isaac
\end{flushright}
\section*{Punto Fijo Omnivariante}
En temas anteriores, cuando se ha ocupado de estudiar procedimientos de resolución aproximada de ecuaciones de la forma:
\begin{center}
    $f(x) = 0$
\end{center}
Continuaremos con la misma tarea, pero con un enfoque diferente, es decir, con ecuaciones del tipo:
\begin{center}
    $x = g(x)$
\end{center}
Lleva poco tiempo darse cuenta de que las expresiones anteriores son de alguna manera equivalentes, ciertamente dada la ecuación $f(x) = 0$ y una solución suya. $P$ existe tal que entonces una función $g()$ (y más de una) tal que la ecuación $x = g(x)$ tiene a $P$ por solución, es decir $P = g(P)$. Y de forma contraria, si $P$ es una solución de $x = g(x)$, entonces $P$ es un cero de la función definida mediante
\begin{center}
    $f(x) = x - g(x)$
\end{center}
Esto nos da pie a dar la siguiente definición.
\section*{Notas Personales}
El coeficiente y la expresión que tiene son quienes le dan la identidad a los sistemas de ecuaciones.
\end{document}